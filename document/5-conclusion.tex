\chapter{結論}
%本論文ではユーザの発話を考慮したユーモアを有する非タスク指向型対話システムを提案した。\\
%\hspace{1zw}提案システムは,ユーザにテキスト入力で対話を行ってもらった。その文から感情や極性,文体を読み取り,予備実験の基,適切なタイミングでかつ,適切な内容でユーモアを発話するようにした。そうすることで,ユーモアの効果を最大限に引き出すようにした。\\
%\hspace{1zw}評価実験の結果から,非タスク指向型対話システムにおけるユーモアの重要性を示し,そのユーモアのタイミングと内容の考慮の有効性を示した。\\
%\hspace{1zw}今後は,ユーザに音声入力を行ってもらい,イントネーションや間などユーザから得る情報を増やす。それらの情報も同時に利用することで,より適切なユーモアのタイミングと内容を推定するようにする。


本論文ではユーザの発話を考慮したユーモアを有する非タスク指向型対話システムを提案した。\\
\hspace{1zw}提案システムでは,ユーザの発話文から,ユーザの感情や極性,またその文体を推定し,それらの基,ユーモアを用いるか否かを決定した。
またユーモアを用いる際は,予めユーモアの内容を,提唱されたユーモアの種類に応じ,細分化されたユーモアの中から決定した。\\
\hspace{1zw}その結果,評価実験から対話システムの性能とユーモアの質の向上が見られた。\\
\hspace{1zw}今後は,入力文の中に,ユーモア生成に必要な情報が無くても,ユーモアの性質を活かしたテンプレートではなく,動的なユーモアを応答するよう心がけたい。
そのためには,ユーザとの会話でユーザに関する情報,つまり知識を獲得し,そのユーザに応じたユーモアの応答を行うようにし,ユーモアの質を上げる。\\
\hspace{1zw}また,発話をテキスト入力ではなく,音声入力に転換し,イントネーションや間などユーザから得る情報を増やす。その情報も利用することで,より適切なユーモアを応答するようにして,システムの向上化を目指す。