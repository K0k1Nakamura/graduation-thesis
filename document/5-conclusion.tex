%!TEX root = ../main.tex
\chapter{結論}
本論文では.IL-RBMを改良し.強化学習タスクへの応用方法を示した.

ニューラルネットワークを用いた強化学習を行うにあたり,ニューラルネットワークにおける有効な追加学習法を考案する必要性があった.

文献\cite{osawa}で提案されたIL-RBMは追加学習が可能であり,与えられたデータが既学習か未学習かを判定することで、自動的に未学習データセットを構築可能であった.またその未学習データセットを用いて追加学習が可能であった.
しかし,IL-RBMはデータの未学習,既学習の判定は可能であるものの,強化学習への応用の際.与えられたデータが正の報酬に関連づくものか.負の報酬に関連づくものかを区別することが出来なかった.

そこで.IL-RBMに正と負の二種類のネットワークを持たせることで.IL-RBMが正の報酬と関連の強いデータと負の報酬と関連の強いデータを区別することが可能であることを示した.

IL-RBMが正のネットワークと負のネットワークを持つことで.既存手法では扱えなかった負のサブゴールを設定可能となり.負の報酬を避けるような長期的な戦略を獲得できることを示した.

また.提案したIL-RBMをもちいたエージェントに三目並べタスクを実際に解かせた.
エージェントは正の報酬へ繋がる行動のデータセットと負の報酬へ繋がる行動のデータセットを採集し.それぞれを提案した正負のネットワークで学習することで.負のネットワークを持たないIL-RBMより高い勝率で勝つことができることを示した.また.負の報酬を避けるような長期的な戦略により.より顕著に敗北率が減少することを示した.

ニューラルネットワークを行う強化学習において、有効な追加学習法を示せたことにより、より抽象度の高い応用例を示すことを今後の課題とする。