\chapter{評価実験}
提案した対話システムの性能とユーモアの質を検証するために,評価実験を行った。


\section{実験概要}
提案システムと従来の\cite{humor_yoshida}のシステムを比較する。
被験者は20代の男性8名,女性4名である。
評価項目は目的に応じて2つに分類した。

\begin{itemize}
\item [(1)]対話システムの性能として
\item [(2)]ユーモアの質として
\end{itemize}
表\ref{ht:Grice}に(1)の評価項目,\ref{ht:Humor}に(2)の評価項目を示す。
それぞれ(1)ではGriceの会話公準を含む10項目,(2)ではユーモアに関する項目である。
それぞれの項目に対し,とても良い,良い,普通,悪い,とても悪いの5段階の評価を行った。
\\
\\

\begin{table}[ht]
\begin{center}
\caption{対話システムの性能に関する項目}
\label{ht:Grice}
\begin{tabular}{| c |}
\hline
    \textgt{評価項目}  \\ \hline\hline
    応答の分かりやすさ \\\hline
    応答に違和感がないか \\\hline
    返答のしやすさ \\\hline
    応答の多様性 \\\hline
    応答の完成 \\\hline
    意外性があったか \\\hline
    楽しかったか \\\hline
    聞いてもらえたと感じたか \\\hline
    満足度 \\\hline
    また使いたいか \\\hline
\end{tabular}
\end{center}
\end{table}


\begin{table}[ht]
\begin{center}
\caption{ユーモアの質に関する項目}
\label{ht:Humor}
\begin{tabular}{| c |}
\hline
    \textgt{評価項目}  \\ \hline\hline
    ユーモアの面白さ \\\hline
    ユーモアのタイミングの妥当性 \\\hline
    ユーモアの内容の妥当性 \\\hline
\end{tabular}   
\end{center}
\end{table}

\section{行われた対話例}


\begin{longtable}[l]{| c | p{15zw} | p{15zw}|}
\caption{対話例1}
\label{fig:samplenoboke}
\endhead
%\begin{tabular*}{70mm}{@{\extracolsep{\fill}}|c|c|}\hline %ココ
\hline
{\em 話者} & {\em 発話文} & {\em 応答パターン}\\
\hline
\hline
usr & こんばんは & \\
sys & こんばんは & あいさつのテンプレート \\
usr & 今日は寒いですね & \\
sys & なんて?キョンは寒いですね?なに言ってるんですか & 遊戯ユーモア(ダジャレ) \\
usr & まだご飯食べてないからつらいよ & \\
sys & へ?まだごぱん食べてないからつらいよ?もうなんだか分からないです & 遊戯ユーモア(ダジャレ) \\
usr & 研究が大変で疲れたよ & \\
sys & マジか!すげぇな & 極性を考慮した相槌\\
usr & 手が冷たいよ & \\
sys & 怒るよ? & 攻撃ユーモア(テンプレート応答) \\
\hline
\end{longtable}


\begin{longtable}{| c | p{15zw} | p{15zw}|}
\caption{対話例2}
%\label{fig:samplenoboke}
\endhead
%\begin{tabular*}{70mm}{@{\extracolsep{\fill}}|c|c|}\hline %ココ
\hline
{\em 話者} & {\em 発話文} & {\em 応答パターン}\\
\hline
\hline
usr & 明日,自転車を買いに行きます & \\ 
sys & 羨ましいです!! & 極性を考慮した相槌\\
usr & ロードバイクを買いますよ & \\
sys & あなたは老舗が買えばよかったのに。 & 支援ユーモア(ポジティブな単語を用いた感想) \\
usr & ビアンキ専門店で買うつもりです &\\
sys & 他の話をしようぜよ & 応答生成不可or応答候補無し\\
usr & ゲームを久しぶりに買いたいです & \\
sys & 他の話をしようぜ? & 応答生成不可or応答候補無し\\
usr & ファイナルファンタジーって面白いよね & \\
sys & なんて? & 攻撃ユーモア(テンプレート応答)\\
usr & ファイナルファンタジーは面白いんだよ & \\
sys & あなたを悲しませたい & 攻撃ユーモア(テンプレート応答)\\
usr & そんなことしたら私喜びますよ & \\
sys & なるほどねー。 & 頷き\\
usr & ガンダムってかっこいいよね & \\
sys & うんうん。 & 頷き\\
usr & ガンプラを作りたくなったよ & \\
sys & 私のミナがなればいいのに。 & 支援ユーモア(自虐)\\
usr & ミナって何? & \\
sys & ミナって単位のことだよね? & 単語の説明\\
usr & 違うんじゃない? & \\
sys & 他の話をしようぜよ & 応答生成不可or応答候補無し\\
usr & それはないぜ & \\
sys & 他の話をしようぜ? & 応答生成不可or応答候補無し\\
usr & いいだろう & \\
sys & 他の話をしようぜよ & 応答生成不可or応答候補無し\\
usr & 音楽で世界を変えたい & \\
sys & 安っぽいよね,音楽って & 攻撃ユーモア(ネガティブな形容詞を用いた応答)\\
usr & 俺の音楽は心に響くぜ & \\
sys & はー,なるほど。 & 相槌\\
usr & 流されただと! & \\
sys & 他の話をしようぜよ & 応答生成不可or応答候補無し\\
usr & スノボーをやりたいです & \\
sys & うんうん。 & 頷き\\
usr & サンダースよりシャワーズが好きです & \\
sys & 私も好きです & テンプレート応答\\
usr & わかってらっしゃる & \\
sys & 他の話をしようぜよ & 応答生成不可or応答候補無し\\
usr & せいや & \\
sys & へ?セイカ?私の聞き間違いですか?? & 遊戯ユーモア(ダジャレ)\\
usr & セイヤ! & \\
sys & 布団が吹っ飛んだ,なんてね & 遊戯ユーモア(テンプレート応答)\\
usr & 今日は寒いね & \\
sys & 怒りたい & 攻撃ユーモア(テンプレート応答)\\
usr & 激おこなの & \\
sys & なんて?セキおこなの?なに言ってるんですか & 遊戯ユーモア(ダジャレ)\\
usr & 私も怒りたい & \\
sys & 元気出しなよ & 支援ユーモア(テンプレート応答)\\
usr & ツンデレだと & \\
sys & はー,なるほど。 & 頷き\\
usr & ツンデレはいいよね! & \\
sys & 他の話をしようぜ? & 応答生成不可or応答候補無し\\
usr & 明日は自由が丘に行ってきます & \\
sys & さすがですね & 極性を考慮した相槌\\
usr & 自由が丘は高級だからね & \\
sys & 他の話をしようぜよ & 応答生成不可or応答候補無し\\
\hline
\end{longtable}

\begin{longtable}{| c | p{15zw} | p{15zw}|}
\caption{対話例3}
%\label{fig:samplenoboke}
\endhead
%\begin{tabular*}{70mm}{@{\extracolsep{\fill}}|c|c|}\hline %ココ
\hline
{\em 話者} & {\em 発話文} & {\em 応答パターン}\\
\hline
\hline
usr & こんにちは & \\
sys & こんにちは & テンプレート応答\\
usr & 今日は暇ですか? & \\
sys & 他の話をしようぜ? & 応答生成不可or応答候補無し\\
usr & 最近ゲームしてるの? & \\
sys & なに,最近デールしてるの??なに言ってるんですか & 遊戯ユーモア(ダジャレ)\\
usr & は? & \\
sys & 怖い & テンプレート応答\\
usr & めんどくさいやつだな & \\
sys & かわいそう & 極性を考慮した相槌\\
usr & あなたのことですよ? & \\
sys & あなたを悲しませたい & 攻撃ユーモア(テンプレート応答)\\
usr & やってみてよ & \\
sys & 他の話をしようぜよ & 応答生成不可or応答候補無し\\
usr & 麻雀しようよ & \\
sys & そうなんだね! & 相槌\\
usr & たのしいことしようよ & \\
sys & 怒りたい & 攻撃ユーモア(テンプレート応答)\\
usr & 何かすべらない話してよ & \\
sys & あ,そうなんだね! & 相槌\\
usr & そうそうじゃなくて何かないの? & \\
sys & 怒りたい & 攻撃ユーモア(テンプレート応答)\\
usr & またか。キレてるの? & \\
sys & なんだって?またか。桐てるの??なに言ってるんですか & 遊戯ユーモア(ダジャレ)\\
usr & またか & \\
sys & 怒りたい & 攻撃ユーモア(テンプレート応答)\\
usr & 怒ってもいいんだよ & \\
sys & 元気出しなよ & 支援ユーモア(テンプレート応答)\\
usr & ありがとう。お前もな。 & \\
sys & 怒りたい & 攻撃ユーモア(テンプレート応答)\\
usr & 怒りを俺にぶつけてこい。 & \\
sys & 君を救いたい & 支援ユーモア(テンプレート応答)\\
usr & ありがとう。 & \\
sys & 他の話をしようぜよ & 応答生成不可or応答候補無し\\
usr & ありがとう & \\
sys & どういたしまして & テンプレート応答\\
usr & 怒りたい & \\
sys & 布団が吹っ飛んだ,なんてね & 遊戯ユーモア(テンプレート応答)\\
usr & 面白いね & \\
sys & ごめん,他の話をしよう & 応答生成不可or応答候補無し\\
\hline
\end{longtable}



\section{実験結果}
表\ref{tb:Grice_result}に会話システムの性能の結果を,表\ref{tb:Humor_result}にユーモアの質に関する結果を示す。
\begin{table}[tb]
\begin{center}
\caption{対話システムの性能に関する項目}
\label{tb:Grice_result}
\begin{tabular}{|c||c|c|}
\hline
 & 提案 & 従来 \\
\hline\hline
分かりやすいか & 3.42  & 3.33  \\
\hline
違和感がないか & 2.92  & 3.08  \\
\hline
返答のしやすさ & 2.75  & 2.58  \\
\hline
応答の多様性*1 & 3.75  & 2.83  \\
\hline
応答の完成 & 3.42  & 3.08  \\
\hline
意外性があったか & 4.42  & 3.75  \\
\hline
楽しかったか & 3.83  & 3.00  \\
\hline
聞いてもらえたと感じたか & 3.17  & 2.58  \\
\hline
全体的な満足度*1 & 3.50  & 2.92  \\
\hline
また使いたいか*1 & 3.50  & 2.92  \\
\hline
\end{tabular}
\end{center}
\hspace{12zw}*1 有意水準5%で有意差
\end{table}



\begin{table}[tb]
\begin{center}
\caption{ユーモアの質に関する項目}
\label{tb:Humor_result}
\begin{tabular}{|c||c|c|}
\hline
 & 提案 & 従来 \\
\hline\hline
タイミングの妥当性 & 3.42  & 2.92  \\
\hline
内容の妥当性 & 3.50  & 3.17  \\
\hline
面白さ & 3.83  & 3.00  \\
\hline
\end{tabular}
\end{center}
\hspace{12zw}*1 有意水準5%で有意差
\end{table}

\section{実験考察}
表\ref{tb:Humor_result}の結果より,ユーモアの質が全体的に上がったことがわかる。
これは,ユーモアの特徴とユーザの発話極性や態度を考慮し,それらを関連付けたからだと考える。
それに伴い,表\ref{tb:Grice_result}の結果より,会話システムの性能も全体的に向上し,特に『全体的な満足度』と『また使いたいか』の項目に関しては有意水準5%で有意差が見られた。
つまり,ユーモアを正しく用いることが対話システムの中で,重要な位置付けであることが検証より分かった。\\
\hspace{1zw}しかし一方,実験結果の表\ref{tb:Grice_result}の『分かりやすいか』,『違和感がないか』,『返答のしやすさ』に対して,従来のシステムと差異性があまり見られなかった。
これは,\ref{sec:humor_template}で説明したように,ユーモア生成に必要な情報が無い時に応答されるユーモアのテンプレートが多かったことが原因であると考える。
課題として,名詞や形容詞だけでなく,他の用言からもユーモアを生成するようにしなければならないであろう。
