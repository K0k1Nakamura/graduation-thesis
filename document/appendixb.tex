\chapter{予備実験で得た結果}
ここでは、予備実験で得たユーモアのタイミングや内容を考慮する材料となった表の例を示す。\\
\hspace{1zw}例1はある男性の結果,例2はある女性の結果である。

\begin{table}
\begin{center}
\caption{極性の例1}
\label{tb:ex1PN}
\begin{tabular}{|c|c|c|p{6em}|p{6em}|p{6em}|}
\hline
\multicolumn{1}{|c}{} & \multicolumn{1}{c}{} & \multicolumn{1}{c|}{} & \multicolumn{3}{c|}{全体的にどの極性が多いか} \\
\cline{4-6}
\multicolumn{1}{|c}{} & \multicolumn{1}{c}{} & \multicolumn{1}{c|}{} & \multicolumn{1}{c|}P & \multicolumn{1}{c|}N & \multicolumn{1}{c|}E \\
\hline
 &  & 攻撃ユーモア & \hspace{2.4zw}0.6 & \hspace{2.4zw}0.2 & \hspace{2.4zw}0.2 \\\cline{3-6}
 & P & 支援ユーモア & \hspace{2.4zw}0.3 & \hspace{2.4zw}0.6 & \hspace{2.4zw}0.5 \\\cline{3-6}
 &  & 遊戯ユーモア & \hspace{2.4zw}0.8 & \hspace{2.4zw}0.4 & \hspace{2.4zw}0.3 \\\cline{3-6}
 &  & ユーモア無し & \hspace{2.4zw}0.2 & \hspace{2.4zw}0.2 & \hspace{2.4zw}0.0 \\\cline{2-6}
 &  & 攻撃ユーモア & \hspace{2.4zw}0.6 & \hspace{2.4zw}0.7 & \hspace{2.4zw}0.0 \\\cline{3-6}
1つ前の & N & 支援ユーモア & \hspace{2.4zw}0.7 & \hspace{2.4zw}0.7 & \hspace{2.4zw}0.5 \\\cline{3-6}
ユーザの &  & 遊戯ユーモア & \hspace{2.4zw}0.5 & \hspace{2.4zw}0.5 & \hspace{2.4zw}0.5 \\\cline{3-6}
発話極性 &  & ユーモア無し & \hspace{2.4zw}0.2 & \hspace{2.4zw}0.2 & \hspace{2.4zw}0.0 \\\cline{2-6}
 &  & 攻撃ユーモア & \hspace{2.4zw}0.6 & \hspace{2.4zw}0.6 & \hspace{2.4zw}0.6 \\\cline{3-6}
 & E & 支援ユーモア & \hspace{2.4zw}0.3 & \hspace{2.4zw}0.4 & \hspace{2.4zw}0.3 \\\cline{3-6}
 &  & 遊戯ユーモア & \hspace{2.4zw}0.7 & \hspace{2.4zw}0.8 & \hspace{2.4zw}0.7 \\\cline{3-6}
 &  & ユーモア無し & \hspace{2.4zw}0.1 & \hspace{2.4zw}0.2 & \hspace{2.4zw}0.2 \\\hline
\end{tabular}
\end{center}
\end{table}


\begin{table}
\begin{center}
\caption{態度の例1}
\label{tb:ex1taido}
\begin{tabular}{|c|c|c|p{6em}|p{6em}|}
\hline
\multicolumn{1}{|c}{} & \multicolumn{1}{c}{} & \multicolumn{1}{c|}{} & \multicolumn{2}{c|}{全体的に態度が良いか否か} \\\cline{4-5}
\multicolumn{1}{|c}{} & \multicolumn{1}{c}{} & \multicolumn{1}{c|}{} & \hspace{2zw}多い & \hspace{1.5zw}少ない \\\hline
 &  & 攻撃ユーモア & \hspace{2.4zw}0.4 & \hspace{2.4zw}0.6 \\\cline{3-5}
 & 良い & 支援ユーモア & \hspace{2.4zw}0.5 & \hspace{2.4zw}0.5 \\\cline{3-5}
 &  & 遊戯ユーモア & \hspace{2.4zw}0.3 & \hspace{2.4zw}0.6 \\\cline{3-5}
 今のユーザの &  & ユーモア無し & \hspace{2.4zw}0.2 & \hspace{2.4zw}0.3 \\\cline{2-5}
態度が &  & 攻撃ユーモア & \hspace{2.4zw}0.5 & \hspace{2.4zw}0.7 \\\cline{3-5}
良いか否か & 悪い & 支援ユーモア & \hspace{2.4zw}0.4 & \hspace{2.4zw}0.3 \\\cline{3-5}
 &  & 遊戯ユーモア & \hspace{2.4zw}0.4 & \hspace{2.4zw}0.5 \\\cline{3-5}
 &  & 無し & \hspace{2.4zw}0.3 & \hspace{2.4zw}0.4 \\\hline
\end{tabular}
\end{center}
\end{table}

\begin{table}[tb]
\begin{center}
\caption{ユーモアの連続性の例1}
\label{tb:ex1humor}
\begin{tabular}{| c | c |}
\hline
     \multicolumn{2}{| c |}{ユーモアの連続性} \\\hline
     攻撃ユーモアの連続性 & 0.3 \\\hline
     支援ユーモアの連続性 & 0.5 \\\hline
	 遊戯ユーモアの連続性 & 0.6 \\\hline
     
\end{tabular}
\end{center}
\end{table}




\begin{table}[tb]
\begin{center}
\caption{重みづけの例1}
\label{tb:ex1weight1}
\begin{tabular}{| c | c |}
\hline
     \multicolumn{2}{| c |}{重みづけ} \\\hline
	 ユーザの態度 & 0.1 \\\hline
     入力文の極性 & 0.5 \\\hline
	 ユーモアの連続性 & 0.4 \\\hline
     
\end{tabular}
\end{center}
\end{table}



\begin{table}
\begin{center}
\caption{極性の例2}
\label{tb:ex1PN}
\begin{tabular}{|c|c|c|p{6em}|p{6em}|p{6em}|}
\hline
\multicolumn{1}{|c}{} & \multicolumn{1}{c}{} & \multicolumn{1}{c|}{} & \multicolumn{3}{c|}{全体的にどの極性が多いか} \\
\cline{4-6}
\multicolumn{1}{|c}{} & \multicolumn{1}{c}{} & \multicolumn{1}{c|}{} & \multicolumn{1}{c|}P & \multicolumn{1}{c|}N & \multicolumn{1}{c|}E \\
\hline
 &  & 攻撃ユーモア & \hspace{2.4zw}0.8 & \hspace{2.4zw}0.2 & \hspace{2.4zw}0.3 \\\cline{3-6}
 & P & 支援ユーモア & \hspace{2.4zw}0.4 & \hspace{2.4zw}0.8 & \hspace{2.4zw}0.5 \\\cline{3-6}
 &  & 遊戯ユーモア & \hspace{2.4zw}0.7 & \hspace{2.4zw}0.5 & \hspace{2.4zw}0.6 \\\cline{3-6}
 &  & ユーモア無し & \hspace{2.4zw}0.4 & \hspace{2.4zw}0.6 & \hspace{2.4zw}0.5 \\\cline{2-6}
 &  & 攻撃ユーモア & \hspace{2.4zw}0.6 & \hspace{2.4zw}0.6 & \hspace{2.4zw}0.5 \\\cline{3-6}
1つ前の & N & 支援ユーモア & \hspace{2.4zw}0.7 & \hspace{2.4zw}0.8 & \hspace{2.4zw}0.6 \\\cline{3-6}
ユーザの&  & 遊戯ユーモア & \hspace{2.4zw}0.5 & \hspace{2.4zw}0.6 & \hspace{2.4zw}0.5 \\\cline{3-6}
発話極性 &  & ユーモア無し & \hspace{2.4zw}0.5 & \hspace{2.4zw}0.3 & \hspace{2.4zw}0.4 \\\cline{2-6}
 &  & 攻撃ユーモア & \hspace{2.4zw}0.7 & \hspace{2.4zw}0.5 & \hspace{2.4zw}0.6 \\\cline{3-6}
 & E & 支援ユーモア & \hspace{2.4zw}0.7 & \hspace{2.4zw}0.8 & \hspace{2.4zw}0.6 \\\cline{3-6}
 &  & 遊戯ユーモア & \hspace{2.4zw}0.5 & \hspace{2.4zw}0.5 & \hspace{2.4zw}0.4 \\\cline{3-6}
 &  & ユーモア無し & \hspace{2.4zw}0.5 & \hspace{2.4zw}0.4 & \hspace{2.4zw}0.6 \\\hline
\end{tabular}
\end{center}
\end{table}


\begin{table}
\begin{center}
\caption{態度の例2}
\label{tb:ex1taido}
\begin{tabular}{|c|c|c|p{6em}|p{6em}|}
\hline
\multicolumn{1}{|c}{} & \multicolumn{1}{c}{} & \multicolumn{1}{c|}{} & \multicolumn{2}{c|}{全体的に態度が良いか否か} \\\cline{4-5}
\multicolumn{1}{|c}{} & \multicolumn{1}{c}{} & \multicolumn{1}{c|}{} & \hspace{2zw}多い &\hspace{1.5zw}少ない \\\hline
 &  & 攻撃ユーモア & \hspace{2.4zw}0.2 & \hspace{2.4zw}0.3 \\\cline{3-5}
 & 良い & 支援ユーモア & \hspace{2.4zw}0.7 & \hspace{2.4zw}0.2 \\\cline{3-5}
 &  & 遊戯ユーモア & \hspace{2.4zw}0.3 & \hspace{2.4zw}0.1 \\\cline{3-5}
 今のユーザの &  & ユーモア無し & \hspace{2.4zw}0.4 & \hspace{2.4zw}0.3 \\\cline{2-5}
態度が &  & 攻撃ユーモア & \hspace{2.4zw}0.3 & \hspace{2.4zw}0.4 \\\cline{3-5}
良いか否か & 悪い & 支援ユーモア & \hspace{2.4zw}0.6 & \hspace{2.4zw}0.3 \\\cline{3-5}
 &  & 遊戯ユーモア & \hspace{2.4zw}0.4 & \hspace{2.4zw}0.2 \\\cline{3-5}
 &  & 無し & \hspace{2.4zw}0.5 & \hspace{2.4zw}0.6 \\\hline
\end{tabular}
\end{center}
\end{table}

\begin{table}[tb]
\begin{center}
\caption{ユーモアの連続性の例2}
\label{tb:ex1humor}
\begin{tabular}{| c | c |}
\hline
     \multicolumn{2}{| c |}{ユーモアの連続性} \\\hline
     攻撃ユーモアの連続性 & 0.2 \\\hline
     支援ユーモアの連続性 & 0.6 \\\hline
	 遊戯ユーモアの連続性 & 0.5 \\\hline
     
\end{tabular}
\end{center}
\end{table}




\begin{table}[tb]
\begin{center}
\caption{重みづけの例2}
\label{tb:ex1weight1}
\begin{tabular}{| c | c |}
\hline
     \multicolumn{2}{| c |}{重みづけ} \\\hline
	 ユーザの態度 & 0.5 \\\hline
     入力文の極性 & 0.3 \\\hline
	 ユーモアの連続性 & 0.2 \\\hline
     
\end{tabular}
\end{center}
\end{table}

