\chapter{システムの概要}
この章では,提案システムの概要を説明していく。
まず,提案システムの流れを図\ref{fig:System}に載せる。

\begin{figure}[tb]
 \begin{center}
  \includegraphics[scale=0.50]{./fig/System.eps}
  \caption{システムの流れ}
  \label{fig:System}
 \end{center}
\end{figure}

この章では特に,『入力文解析』と『応答パターン選択』について説明する。




\section{入力文解析}\label{sec:analysis}
この節では入力文を解析し,解析結果をユーザの極性,態度の推定と2章のユーモア生成に用いる。
まず,極性の推定から説明する。


\subsection{感情による極性推定について}\label{sec:pn}
まずユーザの入力文から感情を推定する方法について述べる。扱う感情の種類としては感情表現辞典
\cite{感情}で定義された10感情を使用した。定義された10感情とは怒り,哀しみ,喜び,好き,厭う,驚き,恥,昂り,安ぎ,恐怖である。
しかし,これらの感情表現辞典を単に用いただけで適切にユーザ主観の感情を厳密に推定することは困難である。例えば,『怒るとすると』というような『怒り+仮定』の場合,ユーザは今の入力に関しては,怒っているとは厳密には言えない。
このように,ただ単に感情表現辞典に含まれるか否かだけでは,適切に感情を推定できないと考える。そこで,入力文に用いられた文法も考慮することにした。
以下,
表\ref{tb:ruleEmotion}に『怒り』を例に文法の種類と,その場合の感情の判断を示す。


\begin{table}[tb]
\begin{center}
\caption{感情と文法による感情推定}
\label{tb:ruleEmotion}
\begin{tabular}{| c | c | c |}
\hline
    \textgt{文法} & \textgt{例} & \textgt{感情の判定} \\ \hline\hline
    現在 & 怒る, 激怒する & ○ \\\hline
    過去 & 怒った, 激怒した & × \\\hline
	否定 & 怒らない & × \\\hline
    自己願望 & 怒りたい, 激怒したい & ○ \\\hline
    他人願望 & 怒って欲しい, 激怒してくれ & × \\\hline
	伝聞推定 & 怒るそうだ, 激怒するらしい & ○ \\\hline
	義務 & 怒るべき, 激怒するべし & ○ \\\hline
	仮定 & 怒るとすると, 激怒するならば & × \\\hline
	命令 & 怒れ, 激怒しろ & × \\\hline
\end{tabular}
\end{center}
\end{table}

ただし,例えば『目も当てられない』などのように,否定『ない』を含んで感情を表現する可能性もある。
このような場合でも,ユーザの感情を正確に抽出するため,感情表現辞典に含まれるか否かの検証を先に行った。

また,感情を直接的には表現しない単語により,感情に関係した表現が行われる場合もある。例えば,『怪我が治る』という入力文の場合,『怪我』や『治る』といった単語は感情表現辞典に記載されていないが,感情として『無感情』でなく『安心』の方が適切である。\\
\hspace{1zw}そこで,高野らが用いた共起の手法を用いる\cite{takano}。この場合だと,『怪我が治る』と前述した10感情との共起を取り,『安心』と判別する。
このようにして,得られた10感情を\ref{sec:PN}で説明する入力文の極性との同一化を図るため,分類する。
まず,Positive(以下Pと略す)の感情として喜び,好き,安心を,Negative(以下Nと略す)の感情として怒り,哀しみ,厭う,恥,恐怖を,の感情として,Neutral(以下Eと略す)驚き,昂りを定義した。ここで,Eの感情として驚きと昂りを定義したのは,この二つの感情はPとNのどちらが適切かの判断がユーザの発話文のみでは難しいためである。\\

\subsection{日本語評価極性辞書による極性推定について}\label{sec:PN}
次に入力文の極性判定について説明する。
入力文における極性を\ref{sec:pn}と同一化を図るため,P,N,Eに区別する。
まず,入力文を形態素解析機CaboCha\cite{cabocha}を用いて,入力文の助詞を除き,単語の原型を得る。例えば,『今日は髪を切った』と入れると,『今日』,『髪』,『切る』といった単語を得ることができる。\\
\hspace{1zw}初めに,このようにして得られた単語が日本語評価極性辞書(以下極性辞書と略す)
\cite{日本語評価}に存在するかどうかを調べる。
極性辞書の特徴として,単語の極性を記載してある辞書のことである。今回は名詞と用言の2種類に極性辞書を分類し,それぞれ別々に極性判定を行う。
表\ref{tb:rulePN}に名詞と用言の極性と文全体の極性のルールを示す。
表\ref{tb:rulePN}は,例えば入力文の名詞がN,用言もNなら入力文はNと判断される。
ここで,複数の名詞が有り,かつそれぞれの極性が異なる場合,極性数が多い方を名詞の極性とする。
例えば,『怪我と病気と褒美』ならばPの数が1,Nの数が2なのでこの場合の名詞の極性はNとなる。同じ数の場合はEとした。用言もこの方式に則る。

\begin{table}[tb]
\begin{center}
\caption{名詞,用言の極性による文の極性判定}
\label{tb:rulePN}
\begin{tabular}{| c | c | c |}
\hline
    \textgt{名詞} & \textgt{用言} & \textgt{極性} \\ \hline\hline
    P & P & P \\\hline
    P & N & N \\\hline
    N & N & N \\\hline
    N & P & P \\\hline
	P & E & P \\\hline
	N & E & N \\\hline
	E & P & P \\\hline
	E & N & N \\\hline
	E & E & E \\\hline
\end{tabular}
\end{center}
\end{table}


この手法による極性の推定の際も,\ref{sec:pn}で用いた文法のルールを適用した。\\
最後に,ここで得られた極性結果と\ref{sec:pn}で得られた極性結果を比べ,同じ場合はその極性を適用し,異なる場合は感情の極性結果を優先した。\\
\hspace{1zw}このようにして、得られた極性を、今の発話の極性,そして会話ログ全体においてどの極性が多いかを調べる。それらを基に
\ref{sec:pre}で得た情報の基,ユーモアの出すタイミングとその内容を決定する。


\subsection{文体について}\label{sec:tai}
ユーモアの出すタイミングや種類の判断の要素となる入力文の文体について説明する。推定方法は,\ref{sec:pn}に述べた,入力文の極性の推定方法と同じである。
この場合の文体とは,入力文に『デス・マス体』が含まれるかどうかのことを指す。
この『デス・マス体』が含まれると,ユーザの態度が丁寧であると判断される。
例えば,『今日は天気が良いです』と『今日は天気が良い』ならば,前者の方がユーザの態度が丁寧と判断される。
このように,今の発話が『デス・マス体』かどうか,そしてユーザの発話ログ全体において、『デス・マス体』が多いか,つまり態度が良いかどうかを判断する。
これらの情報を用いて、表\ref{tb:taido}の基、ユーモアの出すタイミングと内容を決定する。




\section{応答パターン選択}
この節では,システムが発話する応答文の選択方法について説明する。
応答パターンは主に以下の二つである。
\begin{itemize}
\item [(1)]ユーモアを含む応答文
\item [(2)]ユーモアを含まない応答文
\end{itemize}
これらを選ぶ方法として,\ref{sec:timing}で示すアルゴリズムを用いる。
\ref{sec:timing}で選ばれた発話として,(1)が選ばれなければ(2)を返すという手法を用いた。
この他にも,(2)が選ばれる時は, \ref{sec:template}で示したような入力文があった時である






\subsection{ユーモアのタイミングや内容の推定方法}\label{sec:timing}
この節では,\ref{sec:pre}の予備実験で得られた値の基,ユーモアの発話するタイミングと種類の推定方法のアルゴリズムを述べる。
\begin{itemize}

\item [(1)]今,入力された文の極性と態度を\ref{sec:pn}や\ref{sec:tai}で述べた方法で調べる。
\item [(2)]ユーザの発話文全体の,極性と態度を調べる。
\item [(3)]それらの情報の基,表\ref{tb:PN}と表\ref{tb:taido}の数値から,それぞれ妥当なユーモアを候補とする。
この時,極性の提案したユーモアの種類を$H_{PN}$,態度の提案したユーモアの種類を$H_{manner}$とする。
\item [(4)]$H_{PN}$ = $H_{manner}$の場合は,表\ref{tb:renzoku}により,前回使用したユーモア($H_{pre}$とする)を調べる。$H_{PN}$ = $H_{manner}$ ≠ $H_{pre}$の場合は,そのままそのユーモアが選択される。$H_{PN}$ = $H_{manner}$ = $H_{pre}$場合は表\ref{tb:renzoku}の基,
$H_{PN}$(もしくは$H_{manner}$)を選択するか決める。
\item [(5)]$H_{PN}$ ≠ $H_{manner}$ の場合は,表\ref{tb:weight}により,0.57の確率で極性の選択したユーモアが選ばれ, 0.43の確率で態度の選択したユーモアが選ばれる。
\end{itemize}

ここで,具体例を示す。
応答パターンが以下の表\ref{tb:hatuwa}の時を考える。


\begin{table}[tb]
\begin{center}
\caption{A・Bの応答パターン}
\label{tb:hatuwa}
\begin{tabular}{|c|c|c|}
\hline
 & Aの発話 & Bの発話 \\
\hline
直前の極性 & P & N \\
\hline
直前の態度 & デス・マス体 & 無 \\
\hline
全体の極性 & P & N \\
\hline
全体の態度 & デス・マス体 & 無 \\
\hline
前のユーモア & 支援 & 無 \\
\hline
\end{tabular}
\end{center}
\end{table}


まず,表\ref{tb:hatuwa}のAの発話タイプの直前の極性と全体の極性の項と表\ref{tb:PN}より,遊戯ユーモア$>$攻撃ユーモア$>$支援ユーモア$>$ユーモア無の順で$H_{PN}$が選ばれる。\\
\hspace{1zw}また,表\ref{tb:hatuwa}のAの発話タイプの直前の態度と全体の態度の項と表\ref{tb:taido}より,支援ユーモア$>$ユーモア無$>$遊戯ユーモア$>$攻撃ユーモアの順で$H_{manner}$が選ばれる。\\
\hspace{1zw}この時,$H_{PN}$ ≠ $H_{manner}$なら,$H_{PN}$が0.57の確率で優先される。\\
最後に,最終的に選ばれたユーモアが支援ユーモアなら,0.41の確率で支援ユーモアが応答され,0.59の確率で他の応答(支援ユーモア以外)が返される。\\
\hspace{1zw}同じようにBの応答選択率を考える。\\
\hspace{1zw}まず,表\ref{tb:hatuwa}のBの発話タイプの直前の極性と全体の極性の項と表\ref{tb:PN}より,支援ユーモア$>$遊戯ユーモア = ユーモア無$>$攻撃ユーモアの順で$H_{PN}$が選ばれる。\\
\hspace{1zw}また,表\ref{tb:hatuwa}のBの発話タイプの直前の態度と全体の態度の項と表\ref{tb:taido}より,攻撃ユーモア$>$遊戯ユーモア$>$支援ユーモア$>$ユーモア無しの順で$H_{manner}$が選ばれる。\\
\hspace{1zw}ここでAの時と同じように$H_{PN}$と$H_{manner}$が異なる場合,$H_{PN}$が0.57の確率で優先される。\\
\hspace{1zw}そして,最終的に選ばれたユーモアがどれであっても,前に用いたユーモアが無しなので,そのまま選択されたユーモアが選ばれる。\\
\hspace{1zw}まとめると,応答選択率は表\ref{tb:outou}のようになることが期待される。

\begin{table}[tb]
\begin{center}
\caption{A・Bの応答選択率}
\label{tb:outou}
\begin{tabular}{|c|c|c|}
\hline
\raisebox{-1.8ex}[0pt][0pt]& \multicolumn{2}{|c|}{応答選択率} \\
\cline{2-3}
 & A & B \\
\hline
大 & 遊戯 & 支援 \\
\cline{2-3}
\raisebox{-1.8ex}[0pt][0pt]{|} & 支援 & 攻撃 \\
\cline{2-3}
 & 無 & 遊戯 \\
\cline{2-3}
小 & 攻撃 & 無 \\
\hline
\end{tabular}
\end{center}
\end{table}


\subsection{ユーモアを含まない応答文}
この節ではユーモアを含まない応答文について説明する。ユーモアを含まない応答文が選ばれる可能性として以下の2通りがある。
\begin{itemize}
\item [(1)] ユーモアを出